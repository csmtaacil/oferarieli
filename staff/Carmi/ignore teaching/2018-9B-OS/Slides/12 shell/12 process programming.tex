% !Tex root=12 process programming slides.tex
\title[Shell]{Shell}
\subtitle{Process programming}
\author{Carmi Merimovich}
\institute{Tel-Aviv Academic College}
\date{December 11, 2018}
%
\usetheme{Singapore}
\useoutertheme{infolines}
\beamertemplatenavigationsymbolsempty

\usepackage{amsmath}
\everymath{\mathtt{\xdef\tmp{\fam\the\fam\relax}\aftergroup\tmp}}

\usepackage{pgffor,pgfmath}
\usepackage{tikz}
\usetikzlibrary{quotes,angles,positioning,arrows,decorations.pathreplacing,tikzmark}
\usetikzlibrary{arrows.meta,bending}
\usetikzlibrary{shapes}
\usetikzlibrary{automata}

\usepackage{register}
\usepackage{bytefield}

\usepackage{listings}
\lstset{tabsize=1,stepnumber=1000,numberfirstline=true,
										numberstyle=\tiny,xleftmargin=2.5ex,
										language=c,
										morekeywords={uint,ushort,uchar},
										escapeinside={(*}{*)},
										moredelim=**[is][\color{red}]{@}{@}	
										}

\begin{document}
\maketitle
%
%
%
\lstset{tabsize=1,stepnumber=1000,numberfirstline=true,
										numberstyle=\tiny,xleftmargin=2.5ex}
%
%
%
\begin{frame}{{\bf open/pipe/close} system calls}
\begin{enumerate}
\item
	{\tt  int fd = open(char *name, int flags)};
\item
	{\tt int pipe(int fd[2]);}
\item
	{\tt int close(fd);}
\end{enumerate}
\end{frame}
%
%
%
\begin{frame}{The first process}
\begin{itemize}
\item
	The kernel sets the initial state to:
	\begin{itemize}
		\item
				cwd is ``/''.
		\item
			No file is open.
	\end{itemize}
\item
	Sets standard input, standard output, and standard error, to the console device.
\item
	Creates a process to run the shell ({\bf sh}).
\item
	Enters an infinite loop of {\tt \bf wait()}'s.
\end{itemize}
\end{frame}
%
%
%
%
\begin{frame}{}
\vfill
\begin{center}
{\Large
{\bf sh} main functionality \\}
\end{center}
\vfill
\end{frame}
%
%
\begin{frame}[fragile]{{\tt \bf sh} main loop}
\begin{lstlisting}
	while (read(0, cmd, ...) > 0) {
		if (cmd is internal command)
			executeInternalCmd(cmd);
		else
			forkExternalCmd(cmd);
	}
	exit();
\end{lstlisting}
\begin{itemize}
\item
	Internal cmd ``cd'' causes execution of the {\bf \tt chdir} system call.
\item
	External commands are assumed to be executble files.
\end{itemize}
\end{frame}
%
%
\begin{frame}[fragile]{{\tt \bf sh} example: Simple exec}
Typing:
\begin{lstlisting}
ls
\end{lstlisting}
will use the following code, 
	\only<2>{{\color{red}where the parent {\tt\bf sh} executes:}}
	\only<3>{{\color{red}and the child {\tt\bf sh} executes:}}
\begin{onlyenv}<1>
\begin{lstlisting}
	pid = fork();
	if (pid == 0) {
		char *argv[] = {"ls", 0};
		exec("ls", argv);
		exit();
	}
	wait();
\end{lstlisting}
\end{onlyenv}
\begin{onlyenv}<2>
\begin{lstlisting}
	@pid = fork();@
	@if (pid == 0)@ {
		char *argv[] = {"ls", 0};
		exec("ls", argv);
		exit();
	}
	@wait()@;
\end{lstlisting}
\end{onlyenv}
\begin{onlyenv}<3>
\begin{lstlisting}
	@pid =@ fork();
	@if (pid == 0) {@
		@char *argv[] = {"ls", 0};@
		@exec("ls", argv);@
		@exit();@
	}
	wait();
\end{lstlisting}
\end{onlyenv}
\end{frame}
%
%
%
%
\begin{frame}[fragile]{{\tt \bf sh} example: Simple exec}
Typing:
\begin{lstlisting}
ls -l
\end{lstlisting}
will use the code, 
\only<2>{{\color{red}where the parent {\tt\bf sh} executes:}}
	\only<3>{{\color{red}and the child {\tt\bf sh} executes:}}
\begin{onlyenv}<1>
\begin{lstlisting}
	pid = fork();
	if (pid == 0) {
		char *argv[] = {"ls", "-l",0};
		exec("ls", argv);
		exit();
	}
	wait();
\end{lstlisting}
\end{onlyenv}
\begin{onlyenv}<2>
\begin{lstlisting}
	@pid = fork();@
	@if (pid == 0)@ {
		char *argv[] = {"ls", "-l",0};
		exec("ls", argv);
		exit();
	}
	@wait()@;
\end{lstlisting}
\end{onlyenv}
\begin{onlyenv}<3>
\begin{lstlisting}
	@pid = @fork();
	@if (pid == 0) {@
		@char *argv[] = {"ls", "-l",0};@
		@exec("ls", argv);@
		@exit();@
	}
	wait();
\end{lstlisting}
\end{onlyenv}
\end{frame}
%
%
\begin{frame}[fragile]{sh example: Output redirection}
Typing:
\begin{lstlisting}
ls > a.txt
\end{lstlisting}
will use the code, 
	\only<2>{{\color{red}where the parent {\tt\bf sh} executes:}}
	\only<3>{{\color{red}and the child {\tt\bf sh} executes:}}
\begin{onlyenv}<1>
\begin{lstlisting}
	pid = fork();
	if (pid == 0) {
		close (1);
		open("a.txt", O_CREAT);
		char *argv[] = {"ls", 0};
		exec("ls", argv);
		exit();
	}
	wait();
\end{lstlisting}
\end{onlyenv}
\begin{onlyenv}<2>
\begin{lstlisting}
	@pid = fork();@
	@if (pid == 0) @{
		close (1);
		open("a.txt", O_CREAT);
		char *argv[] = {"ls", 0};
		exec("ls", argv);
		exit();
	}
	@wait();@
\end{lstlisting}
\end{onlyenv}
\begin{onlyenv}<3>
\begin{lstlisting}
	@pid = @fork();
	@if (pid == 0) {@
		@close (1);@
		@open("a.txt", O_CREAT);@
		@char *argv[] = {"ls", 0};@
		@exec("ls", argv);@
		@exit();@
	}
	wait();
\end{lstlisting}
\end{onlyenv}
\end{frame}
%
%
%
%
\begin{frame}[fragile]{sh example: Output redirection}
Typing:
\begin{lstlisting}
ls -l > b.txt
\end{lstlisting}
will use the code, 
	\only<2>{{\color{red}where the parent {\tt\bf sh} executes:}}
	\only<3>{{\color{red}and the child {\tt\bf sh} executes:}}
\begin{onlyenv}<1>
\begin{lstlisting}
	pid = fork();
	if (pid == 0) {
		close (1);
		open("b.txt", O_CREAT);
		char *argv[] = {"ls", "-l", 0};
		exec("ls", argv);
		exit();
	}
	wait();
\end{lstlisting}
\end{onlyenv}
\begin{onlyenv}<2>
\begin{lstlisting}
	@pid = fork();@
	@if (pid == 0)@ {
		close (1);
		open("b.txt", O_CREAT);
		char *argv[] = {"ls", "-l", 0};
		exec("ls", argv);
		exit();
	}
	@wait();@
\end{lstlisting}
\end{onlyenv}
\begin{onlyenv}<3>
\begin{lstlisting}
	@pid = @fork();
	@if (pid == 0) {@
		@close (1);@
		@open("b.txt", O_CREAT);@
		@char *argv[] = {"ls", "-l", 0};@
		@exec("ls", argv);@
		@exit();@
	}
	wait();
\end{lstlisting}
\end{onlyenv}
\end{frame}
%
%
\begin{frame}[fragile]{sh example: Input redirection}
Typing:
\begin{lstlisting}
sh < b.txt
\end{lstlisting}
will use the code, 
	\only<2>{{\color{red}where the parent {\tt\bf sh} executes:}}
	\only<3>{{\color{red}and the child {\tt\bf sh} executes:}}
\begin{onlyenv}<1>
\begin{lstlisting}
	pid = fork();
	if (pid == 0) {
		close (0);
		open("b.txt", O_RONLY);
		char *argv[] = {"sh", 0};
		exec("sh", argv);
		exit();
	}
	wait();
\end{lstlisting}
\end{onlyenv}
\begin{onlyenv}<2>
\begin{lstlisting}
	@pid = fork();
	if (pid == 0)@ {
		close (0);
		open("b.txt", O_RONLY);
		char *argv[] = {"sh", 0};
		exec("sh", argv);
		exit();
	}
	@wait();@
\end{lstlisting}
\end{onlyenv}
\begin{onlyenv}<3>
\begin{lstlisting}
	@pid = @ fork();
	@if (pid == 0) {
		close (0);
		open("b.txt", O_RONLY);
		char *argv[] = {"sh", 0};
		exec("sh", argv);
		exit();
	}@
	wait();
\end{lstlisting}
\end{onlyenv}
\end{frame}
%
%
\begin{frame}[fragile]{sh example: Pipe}
Typing:
\begin{lstlisting}
cat a.bat | sh
\end{lstlisting}
will use the code:
	\only<2>{{\color{red}where the parent {\tt\bf sh} executes:}}
	\only<3>{{\color{red}the first child {\tt\bf sh} executes:}}
	\only<4>{{\color{red}the second child {\tt\bf sh} executes:}}
\par\noindent\small
\begin{onlyenv}<1>
\begin{minipage}[t]{2.3in}
\begin{lstlisting}[frame=single]
	int p[2];
	pipe(p);
	pid = fork();
	if (pid == 0) {
		close (1);
		dup(p[1]);
		close(p[0]);
		close(p[1]);
		char *argv[] = {"cat", 0};
		exec("cat", argv);
		exit();
	}
\end{lstlisting}
\end{minipage}
\begin{minipage}[t]{2.3in}
\begin{lstlisting}[frame=single]
	pid = fork()
	if (pid == 0) {
		close(0);
		dup(p[0]);
		close(p[0]);
		close(p[1]);
		char *argv[] = {"sh", 0};
		exec("sh", argv);
		exit();	
	}
	close(p[0]);
	close(p[1]);
	wait();
	wait();
\end{lstlisting}
\end{minipage}
\end{onlyenv}
\begin{onlyenv}<2>
\begin{minipage}[t]{2.3in}
\begin{lstlisting}[frame=single]
	int p[2];
	@pipe(p);
	pid = fork();
	if (pid == 0) @{
		close (1);
		dup(p[1]);
		close(p[0]);
		close(p[1]);
		char *argv[] = {"cat", 0};
		exec("cat", argv);
		exit();
	}
\end{lstlisting}
\end{minipage}
\begin{minipage}[t]{2.3in}
\begin{lstlisting}[frame=single]
	@pid = fork()
	if (pid == 0) @{
		close(0);
		dup(p[0]);
		close(p[0]);
		close(p[1]);
		char *argv[] = {"sh", 0};
		exec("sh", argv);
		exit();	
	}
	@close(p[0]);
	close(p[1]);
	wait();
	wait();@
\end{lstlisting}
\end{minipage}
\end{onlyenv}
\begin{onlyenv}<3>
\begin{minipage}[t]{2.2in}
\begin{lstlisting}[frame=single]
	int p[2];
	pipe(p);
	@pid =@ fork();
	@if (pid == 0) {
		close (1);
		dup(p[1]);
		close(p[0]);
		close(p[1]);
		char *argv[] = {"cat", 0};
		exec("cat", argv);
		exit();
	}@
\end{lstlisting}
\end{minipage}
\begin{minipage}[t]{2.3in}
\begin{lstlisting}[frame=single]
	pid = fork()
	if (pid == 0) {
		close(0);
		dup(p[0]);
		close(p[0]);
		close(p[1]);
		char *argv[] = {"sh", 0};
		exec("sh", argv);
		exit();	
	}
	close(p[0]);
	close(p[1]);
	wait();
	wait();
\end{lstlisting}
\end{minipage}
\end{onlyenv}
\begin{onlyenv}<4>
\begin{minipage}[t]{2.2in}
\begin{lstlisting}[frame=single]
	int p[2];
	pipe(p);
	pid = fork();
	if (pid == 0) {
		close (1);
		dup(p[1]);
		close(p[0]);
		close(p[1]);
		char *argv[] = {"cat", 0};
		exec("cat", argv);
		exit();
	}
\end{lstlisting}
\end{minipage}
\begin{minipage}[t]{2.3in}
\begin{lstlisting}[frame=single]
	@pid =@ fork()
	@if (pid == 0) {
		close(0);
		dup(p[0]);
		close(p[0]);
		close(p[1]);
		char *argv[] = {"sh", 0};
		exec("sh", argv);
		exit();@	
	}
	close(p[0]);
	close(p[1]);
	wait();
	wait();
\end{lstlisting}
\end{minipage}
\end{onlyenv}
\end{frame}
%
%
\end{document}